\documentclass{article}
\usepackage{amsmath}
\usepackage{amssymb}
\usepackage[margin=1in]{geometry}
\usepackage{setspace}

\title{The Equitable Financial Inclusion Score (EFIS): A Microeconomic Framework for Credit Assessment in Underserved Populations}
\author{Maya Patel\thanks{Department of Economics, Stanford University. Email: mpatel@stanford.edu. The author thanks the Project Umoja research team for their collaboration and insights. All errors remain the author's own.}}
\date{September 20, 2025}

\begin{document}

\maketitle

\doublespacing

\begin{abstract}
Over 1.4 billion adults lack access to formal credit. Traditional credit scoring models often exclude Asset-Limited, Income-Constrained, Employed (ALICE) individuals by relying on formal financial histories that are unavailable to this demographic. This paper introduces the Equitable Financial Inclusion Score (EFIS), a new methodology for measuring creditworthiness that moves beyond traditional metrics. The EFIS model integrates four components: human capital assessment using Kalman filtering, social capital derived from community networks, reputation signaling through a verified repayment history, and behavioral indicators of financial discipline. The model is designed to address information asymmetries in credit markets and provides a transparent, mathematically rigorous framework for financial inclusion. Preliminary simulations suggest EFIS could expand credit access to previously excluded populations without a significant increase in default risk.
\end{abstract}

\textbf{JEL Classification:} D82, G21, O16, C58

\textbf{Keywords:} financial inclusion, credit scoring, information asymmetry, microfinance, behavioral economics

\section{Introduction}

A persistent market failure in global finance is the exclusion of creditworthy individuals who lack formal financial histories. Lenders seek reliable borrowers, and many unbanked or underbanked individuals would repay loans if given the opportunity. The inability to connect these parties stems from information asymmetry; conventional risk assessment tools are unable to measure the creditworthiness of applicants who operate outside the formal economy.

This paper proposes a solution that re-evaluates the measurement of credit risk by focusing on economic potential rather than solely on financial history, incorporating community relationships, and recognizing creditworthiness as a dynamic attribute that evolves through performance and reputation. The Equitable Financial Inclusion Score (EFIS) formalizes these concepts into a quantitative framework. The model synthesizes principles from multiple economic fields, using stochastic calculus to filter income data, network theory to quantify social capital, game theory to model reputation, and behavioral economics to assess financial discipline.

The scale of financial exclusion is substantial. The World Bank (2021) estimates that 1.4 billion adults are unbanked. This figure does not account for the many individuals who have access to basic banking services but are denied credit, forcing them to rely on informal lenders who often charge usurious interest rates. Such exclusion has significant economic consequences, preventing individuals from smoothing consumption, investing in assets, or expanding businesses. These microeconomic constraints contribute to macroeconomic inefficiency, as countries with greater financial inclusion often exhibit higher GDP growth and lower inequality.

Traditional credit scoring systems, such as the FICO score developed in 1989, function effectively in economies with well-developed financial infrastructure and populations with extensive credit histories. These models fail in developing economies where such conditions are not met. The core issue is information asymmetry: without formal signals of credit risk, lenders cannot effectively distinguish between low-risk and high-risk borrowers. This can lead to adverse selection, where high average interest rates drive creditworthy borrowers from the market.

Microfinance emerged as a response to this problem, using alternative information sources like peer monitoring in group lending models. While these innovations have expanded credit access for millions, microfinance has limitations, including high interest rates, small loan sizes, and a primary focus on rural areas with dense social networks. More recently, financial technology firms have utilized alternative data, such as mobile phone usage or social media activity, to assess credit risk. These methods can be effective but risk replicating existing biases, often lack transparency, and are limited to individuals with a significant digital footprint.

EFIS differs from these approaches by reconceptualizing creditworthiness as a dynamic and multidimensional attribute. The framework treats credit risk as a process, incorporates social networks as a form of collateral, and provides a mechanism for individuals to build a reputation from a zero-history starting point. The human capital component is informed by research showing that skills and education are strong predictors of earnings. The social capital component applies network economics to demonstrate how relationships generate economic value. The reputation element uses game-theoretic models of repeated interaction, and the behavioral component leverages findings from experimental economics on time preferences and decision-making.

The model is designed for transparency. Unlike proprietary credit scoring algorithms, each component of EFIS has a clear mathematical formulation, allowing users to understand how their score is calculated and identify pathways for improvement. The system is architected to operate on a self-sovereign identity platform, where individuals control their own data and choose what information to share. This structure protects user privacy and reduces the risks associated with centralized data repositories.

This paper is structured as follows. Section 2 presents the mathematical specification of the EFIS model. Section 3 discusses the theoretical foundations for each component. Section 4 concludes with policy implications and directions for future research.

\section{The EFIS Model Architecture}

The Equitable Financial Inclusion Score for individual $i$ at time $t$ is defined as a weighted sum of four components:
$$
\mathcal{U}_i(t) = \omega_H \cdot H_i(t) + \omega_S \cdot S_i(t) + \omega_R \cdot R_i(t) + \omega_B \cdot B_i(t)
$$
Each component $k$ is assigned a weight $\omega_k$ reflecting its predictive power, with $\sum \omega_k = 1$. The weights are calibrated using machine learning techniques on historical repayment data. The resulting score is scaled to a range of 0 to 1000 to be familiar to market participants.

\subsection{Human Capital Assessment ($H_i(t)$): Signal Extraction from Noisy Income Data}

Conventional credit scoring requires a stable income history, which disadvantages individuals with viable but volatile income streams, such as street vendors, seasonal farmers, or gig economy workers. To address this, EFIS employs a state-space model to distinguish an individual's underlying economic capacity from temporary income fluctuations. True economic capacity is modeled as an unobserved state variable, $\theta_t$, while observed income, $y_t$, is a noisy measurement of this state:
$$
\theta_t = \phi \theta_{t-1} + \epsilon_t
$$
$$
y_t = \theta_t + \nu_t
$$
The state equation models the evolution of true capacity with persistence $\phi$ and an innovation term $\epsilon_t$. The observation equation relates measured income to true capacity with measurement error $\nu_t$. The Kalman filter is used to derive an optimal estimate of $\theta_t$ based on the history of income observations, accounting for both the level and volatility of earnings.

The human capital score also incorporates verified credentials, such as education certificates and professional licenses:
$$
H_i(t) = \left( \mathbb{E}[\theta_t | y_{1:t}] \right) \cdot \left(1 + \sum_{j=1}^{n} \delta_j c_{ij} \right)
$$
Credentials $c_{ij}$ are assigned weights $\delta_j$ based on their empirical correlation with future earnings, acknowledging that income-generating capacity extends beyond recent earnings.

\subsection{Social Capital Quantification ($S_i(t)$): Community Networks as Collateral}

In traditional lending, physical collateral mitigates default risk and incentivizes repayment. Many financially excluded individuals lack such assets but possess valuable social capital through community relationships. EFIS quantifies this social capital using network analysis. An individual's social capital score is a function of endorsements from their verified network connections:
$$
S_i(t) = \sum_{j \in N_i} \frac{v_{ji} \cdot \mathcal{U}_j(t-1)}{d(i,j)}
$$
An endorsement $v_{ji}$ from a network member $j$ is weighted by the endorser's own EFIS score from the previous period, $\mathcal{U}_j(t-1)$, and inversely by the network distance $d(i,j)$ between the two individuals. Closer relationships are given greater weight, reflecting stronger informational signals and reciprocal obligations. This formulation incentivizes high-scoring individuals to endorse other reliable borrowers, creating a community-based screening mechanism. The recursive structure ensures that social capital grows as individuals within the network build their own creditworthiness.

\subsection{Reputation Mechanisms ($R_i(t)$): Costly Signaling Through Performance History}

Asymmetric information can be overcome through costly signaling, where high-quality types take actions that low-quality types cannot afford to mimic. EFIS establishes a reputation signaling mechanism for credit markets. Each successful loan repayment generates a non-transferable \textit{Reputation Token} linked to the borrower's digital identity. These public and verifiable tokens serve as a costly signal of reliability.
$$
R_i(t) = \sum_{k=1}^{m} \lambda^k \cdot \tanh\left(\frac{L_k}{C}\right) \cdot \mathbb{I}(\text{repaid}_k)
$$
The reputation score, $R_i(t)$, is a sum over all $m$ previous loans. The contribution of each loan depends on its relative size $L_k/C$, where $C$ is a scaling constant. The indicator function $\mathbb{I}(\text{repaid}_k)$ is 1 for successful repayment and 0 otherwise. The parameter $\lambda > 1$ creates \textit{streaking bonuses} for consecutive repayments, rewarding consistency and allowing new borrowers to build a reputation more rapidly. This exponential structure accelerates reputation growth for reliable borrowers.

\subsection{Behavioral Economic Indicators ($B_i(t)$): Time Preferences and Financial Discipline}

Creditworthiness depends on both the ability and the willingness to repay. While traditional metrics focus on ability, EFIS incorporates behavioral indicators to measure financial discipline and long-term orientation. The model uses observed financial behaviors to estimate an individual's time discount rate, $\hat{\beta}_i$, and loss aversion, $\hat{\gamma}_i$:
$$
B_i(t) = \frac{k_1}{\log(1+\hat{\beta}_i)} + k_2 \hat{\gamma}_i
$$
Time discount rates are inferred from savings patterns and participation in financial education programs. Individuals who save consistently or complete financial literacy courses demonstrate a lower discount rate and a greater capacity for planning. Loss aversion is estimated from behaviors such as purchasing insurance or diversifying income sources, which indicate an awareness of downside risk that correlates with diligence in meeting repayment obligations. The functional form rewards patience (low $\hat{\beta}_i$) and prudent risk management (appropriate $\hat{\gamma}_i$).

\section{Theoretical Foundations and Economic Intuition}

Each EFIS component is grounded in established economic theory, adapted to address the challenges of financial inclusion. The framework as a whole is designed to provide a comprehensive measure of creditworthiness.

\subsection{Information Economics and Market Design}

The primary problem EFIS confronts is adverse selection. When lenders cannot distinguish between borrower types, they must price for average risk, which can drive low-risk borrowers from the market. EFIS introduces alternative screening mechanisms that are incentive-compatible for populations without formal credit histories. The social capital component functions as a screening device based on community information; endorsements represent a social capital investment by the endorser, creating an incentive to screen applicants carefully. The reputation component establishes a separating equilibrium, where creditworthy borrowers signal their type through consistent repayment. The costly signaling structure ensures that only reliable borrowers find it optimal to maintain a perfect repayment record.

\subsection{Network Economics and Social Capital Theory}

The social capital component operationalizes insights from Coleman (1988) on the economic value of community relationships and Granovetter (1985) on the role of social networks in reducing transaction costs. The model's distance weighting reflects the distinction between strong and weak ties, where strong ties (low network distance) provide more reliable information and stronger enforcement mechanisms. The recursive scoring structure captures the relational nature of social capital, as an endorsement from a high-scoring, well-connected individual is more valuable than one from a low-scoring, isolated individual. This can create positive feedback loops that enhance a community's collective creditworthiness.

\subsection{Game Theory and Reputation Mechanisms}

The reputation component draws on game-theoretic models of repeated interactions. In such models, cooperation can be sustained if players value future payoffs sufficiently. EFIS creates analogous incentives with its token system. A default results in the forfeiture of all future reputation-building opportunities. The exponential weighting of consecutive repayments means that the value of maintaining a good reputation grows over time, making default increasingly costly. The streaking bonus addresses the "cold start" problem common in reputation systems by allowing new borrowers to build a valuable reputation quickly through consistent repayment of smaller loans, creating a clear path from exclusion to full credit market participation.

\section{Conclusion}

The EFIS model offers a new framework for credit assessment tailored to the needs of financially excluded populations. By starting from the first principles of creditworthiness instead of adapting existing models, it provides a system that can potentially expand financial inclusion while adhering to prudent risk management.

The model's transparent and explainable structure facilitates real-world testing and builds trust with users and regulators. The definitive test of its efficacy will be its ability to predict loan repayment more accurately than current alternatives. Furthermore, EFIS conceives of financial inclusion as a dynamic process. It provides individuals with clear pathways to improve their scores through education, responsible financial behavior, and community engagement. This approach benefits not only the individual borrower but also the broader economy.

If models like EFIS prove successful, they could help allocate capital more efficiently, unlocking economic potential that is currently constrained by credit market failures. Future research should focus on piloting the EFIS framework in diverse economic settings, calibrating the model's parameters with real-world repayment data, and analyzing its broader macroeconomic impacts. A financial system that evaluates potential alongside history and community alongside collateral can create opportunities for inclusive and sustainable economic growth.

\section{TL;DR}

Where:
\begin{enumerate}
    \item $\mathcal{U}_i(t)$: The Equitable Financial Inclusion Score for individual $i$ at time $t$.
    \item $\omega_H, \omega_S, \omega_R, \omega_B$: Predictive weights for the Human Capital, Social Capital, Reputation, and Behavioral components, respectively, where $\sum \omega = 1$.
    \item $\mathbb{E}[\theta_t | y_{1:t}]$: The expected value of true economic capacity ($\theta_t$) at time $t$, given the history of observed income ($y_{1:t}$), as estimated by a Kalman filter.
    \item $c_{ij}$: A binary variable indicating if individual $i$ possesses verified credential $j$.
    \item $\delta_j$: The weight assigned to credential $j$ based on its correlation with earning potential.
    \item $N_i$: The set of individuals in individual $i$'s verified social network.
    \item $v_{ji}$: A binary variable indicating an endorsement of individual $i$ from network member $j$.
    \item $\mathcal{U}_j(t-1)$: The EFIS score of the endorsing individual $j$ from the previous time period.
    \item $d(i,j)$: The network distance between individuals $i$ and $j$.
    \item $\lambda$: The "streaking bonus" parameter ($\lambda > 1$) for consecutive successful repayments.
    \item $L_k$: The principal size of the $k$-th loan.
    \item $C$: A scaling constant for loan size.
    \item $\mathbb{I}(\text{repaid}_k)$: An indicator function that equals 1 if the $k$-th loan was successfully repaid, and 0 otherwise.
    \item $\hat{\beta}_i$: The estimated time discount rate for individual $i$.
    \item $\hat{\gamma}_i$: The estimated loss aversion parameter for individual $i$.
    \item $k_1, k_2$: Scaling constants for the behavioral components.
\end{enumerate}

\end{document}